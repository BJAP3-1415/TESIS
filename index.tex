% Options for packages loaded elsewhere
\PassOptionsToPackage{unicode}{hyperref}
\PassOptionsToPackage{hyphens}{url}
\PassOptionsToPackage{dvipsnames,svgnames,x11names}{xcolor}
%
\documentclass[
  letterpaper,
  DIV=11,
  numbers=noendperiod]{scrreprt}

\usepackage{amsmath,amssymb}
\usepackage{iftex}
\ifPDFTeX
  \usepackage[T1]{fontenc}
  \usepackage[utf8]{inputenc}
  \usepackage{textcomp} % provide euro and other symbols
\else % if luatex or xetex
  \usepackage{unicode-math}
  \defaultfontfeatures{Scale=MatchLowercase}
  \defaultfontfeatures[\rmfamily]{Ligatures=TeX,Scale=1}
\fi
\usepackage{lmodern}
\ifPDFTeX\else  
    % xetex/luatex font selection
\fi
% Use upquote if available, for straight quotes in verbatim environments
\IfFileExists{upquote.sty}{\usepackage{upquote}}{}
\IfFileExists{microtype.sty}{% use microtype if available
  \usepackage[]{microtype}
  \UseMicrotypeSet[protrusion]{basicmath} % disable protrusion for tt fonts
}{}
\makeatletter
\@ifundefined{KOMAClassName}{% if non-KOMA class
  \IfFileExists{parskip.sty}{%
    \usepackage{parskip}
  }{% else
    \setlength{\parindent}{0pt}
    \setlength{\parskip}{6pt plus 2pt minus 1pt}}
}{% if KOMA class
  \KOMAoptions{parskip=half}}
\makeatother
\usepackage{xcolor}
\setlength{\emergencystretch}{3em} % prevent overfull lines
\setcounter{secnumdepth}{5}
% Make \paragraph and \subparagraph free-standing
\ifx\paragraph\undefined\else
  \let\oldparagraph\paragraph
  \renewcommand{\paragraph}[1]{\oldparagraph{#1}\mbox{}}
\fi
\ifx\subparagraph\undefined\else
  \let\oldsubparagraph\subparagraph
  \renewcommand{\subparagraph}[1]{\oldsubparagraph{#1}\mbox{}}
\fi

\usepackage{color}
\usepackage{fancyvrb}
\newcommand{\VerbBar}{|}
\newcommand{\VERB}{\Verb[commandchars=\\\{\}]}
\DefineVerbatimEnvironment{Highlighting}{Verbatim}{commandchars=\\\{\}}
% Add ',fontsize=\small' for more characters per line
\usepackage{framed}
\definecolor{shadecolor}{RGB}{241,243,245}
\newenvironment{Shaded}{\begin{snugshade}}{\end{snugshade}}
\newcommand{\AlertTok}[1]{\textcolor[rgb]{0.68,0.00,0.00}{#1}}
\newcommand{\AnnotationTok}[1]{\textcolor[rgb]{0.37,0.37,0.37}{#1}}
\newcommand{\AttributeTok}[1]{\textcolor[rgb]{0.40,0.45,0.13}{#1}}
\newcommand{\BaseNTok}[1]{\textcolor[rgb]{0.68,0.00,0.00}{#1}}
\newcommand{\BuiltInTok}[1]{\textcolor[rgb]{0.00,0.23,0.31}{#1}}
\newcommand{\CharTok}[1]{\textcolor[rgb]{0.13,0.47,0.30}{#1}}
\newcommand{\CommentTok}[1]{\textcolor[rgb]{0.37,0.37,0.37}{#1}}
\newcommand{\CommentVarTok}[1]{\textcolor[rgb]{0.37,0.37,0.37}{\textit{#1}}}
\newcommand{\ConstantTok}[1]{\textcolor[rgb]{0.56,0.35,0.01}{#1}}
\newcommand{\ControlFlowTok}[1]{\textcolor[rgb]{0.00,0.23,0.31}{#1}}
\newcommand{\DataTypeTok}[1]{\textcolor[rgb]{0.68,0.00,0.00}{#1}}
\newcommand{\DecValTok}[1]{\textcolor[rgb]{0.68,0.00,0.00}{#1}}
\newcommand{\DocumentationTok}[1]{\textcolor[rgb]{0.37,0.37,0.37}{\textit{#1}}}
\newcommand{\ErrorTok}[1]{\textcolor[rgb]{0.68,0.00,0.00}{#1}}
\newcommand{\ExtensionTok}[1]{\textcolor[rgb]{0.00,0.23,0.31}{#1}}
\newcommand{\FloatTok}[1]{\textcolor[rgb]{0.68,0.00,0.00}{#1}}
\newcommand{\FunctionTok}[1]{\textcolor[rgb]{0.28,0.35,0.67}{#1}}
\newcommand{\ImportTok}[1]{\textcolor[rgb]{0.00,0.46,0.62}{#1}}
\newcommand{\InformationTok}[1]{\textcolor[rgb]{0.37,0.37,0.37}{#1}}
\newcommand{\KeywordTok}[1]{\textcolor[rgb]{0.00,0.23,0.31}{#1}}
\newcommand{\NormalTok}[1]{\textcolor[rgb]{0.00,0.23,0.31}{#1}}
\newcommand{\OperatorTok}[1]{\textcolor[rgb]{0.37,0.37,0.37}{#1}}
\newcommand{\OtherTok}[1]{\textcolor[rgb]{0.00,0.23,0.31}{#1}}
\newcommand{\PreprocessorTok}[1]{\textcolor[rgb]{0.68,0.00,0.00}{#1}}
\newcommand{\RegionMarkerTok}[1]{\textcolor[rgb]{0.00,0.23,0.31}{#1}}
\newcommand{\SpecialCharTok}[1]{\textcolor[rgb]{0.37,0.37,0.37}{#1}}
\newcommand{\SpecialStringTok}[1]{\textcolor[rgb]{0.13,0.47,0.30}{#1}}
\newcommand{\StringTok}[1]{\textcolor[rgb]{0.13,0.47,0.30}{#1}}
\newcommand{\VariableTok}[1]{\textcolor[rgb]{0.07,0.07,0.07}{#1}}
\newcommand{\VerbatimStringTok}[1]{\textcolor[rgb]{0.13,0.47,0.30}{#1}}
\newcommand{\WarningTok}[1]{\textcolor[rgb]{0.37,0.37,0.37}{\textit{#1}}}

\providecommand{\tightlist}{%
  \setlength{\itemsep}{0pt}\setlength{\parskip}{0pt}}\usepackage{longtable,booktabs,array}
\usepackage{calc} % for calculating minipage widths
% Correct order of tables after \paragraph or \subparagraph
\usepackage{etoolbox}
\makeatletter
\patchcmd\longtable{\par}{\if@noskipsec\mbox{}\fi\par}{}{}
\makeatother
% Allow footnotes in longtable head/foot
\IfFileExists{footnotehyper.sty}{\usepackage{footnotehyper}}{\usepackage{footnote}}
\makesavenoteenv{longtable}
\usepackage{graphicx}
\makeatletter
\def\maxwidth{\ifdim\Gin@nat@width>\linewidth\linewidth\else\Gin@nat@width\fi}
\def\maxheight{\ifdim\Gin@nat@height>\textheight\textheight\else\Gin@nat@height\fi}
\makeatother
% Scale images if necessary, so that they will not overflow the page
% margins by default, and it is still possible to overwrite the defaults
% using explicit options in \includegraphics[width, height, ...]{}
\setkeys{Gin}{width=\maxwidth,height=\maxheight,keepaspectratio}
% Set default figure placement to htbp
\makeatletter
\def\fps@figure{htbp}
\makeatother
\newlength{\cslhangindent}
\setlength{\cslhangindent}{1.5em}
\newlength{\csllabelwidth}
\setlength{\csllabelwidth}{3em}
\newlength{\cslentryspacingunit} % times entry-spacing
\setlength{\cslentryspacingunit}{\parskip}
\newenvironment{CSLReferences}[2] % #1 hanging-ident, #2 entry spacing
 {% don't indent paragraphs
  \setlength{\parindent}{0pt}
  % turn on hanging indent if param 1 is 1
  \ifodd #1
  \let\oldpar\par
  \def\par{\hangindent=\cslhangindent\oldpar}
  \fi
  % set entry spacing
  \setlength{\parskip}{#2\cslentryspacingunit}
 }%
 {}
\usepackage{calc}
\newcommand{\CSLBlock}[1]{#1\hfill\break}
\newcommand{\CSLLeftMargin}[1]{\parbox[t]{\csllabelwidth}{#1}}
\newcommand{\CSLRightInline}[1]{\parbox[t]{\linewidth - \csllabelwidth}{#1}\break}
\newcommand{\CSLIndent}[1]{\hspace{\cslhangindent}#1}

\KOMAoption{captions}{tableheading}
\makeatletter
\@ifpackageloaded{tcolorbox}{}{\usepackage[skins,breakable]{tcolorbox}}
\@ifpackageloaded{fontawesome5}{}{\usepackage{fontawesome5}}
\definecolor{quarto-callout-color}{HTML}{909090}
\definecolor{quarto-callout-note-color}{HTML}{0758E5}
\definecolor{quarto-callout-important-color}{HTML}{CC1914}
\definecolor{quarto-callout-warning-color}{HTML}{EB9113}
\definecolor{quarto-callout-tip-color}{HTML}{00A047}
\definecolor{quarto-callout-caution-color}{HTML}{FC5300}
\definecolor{quarto-callout-color-frame}{HTML}{acacac}
\definecolor{quarto-callout-note-color-frame}{HTML}{4582ec}
\definecolor{quarto-callout-important-color-frame}{HTML}{d9534f}
\definecolor{quarto-callout-warning-color-frame}{HTML}{f0ad4e}
\definecolor{quarto-callout-tip-color-frame}{HTML}{02b875}
\definecolor{quarto-callout-caution-color-frame}{HTML}{fd7e14}
\makeatother
\makeatletter
\makeatother
\makeatletter
\@ifpackageloaded{bookmark}{}{\usepackage{bookmark}}
\makeatother
\makeatletter
\@ifpackageloaded{caption}{}{\usepackage{caption}}
\AtBeginDocument{%
\ifdefined\contentsname
  \renewcommand*\contentsname{Table of contents}
\else
  \newcommand\contentsname{Table of contents}
\fi
\ifdefined\listfigurename
  \renewcommand*\listfigurename{List of Figures}
\else
  \newcommand\listfigurename{List of Figures}
\fi
\ifdefined\listtablename
  \renewcommand*\listtablename{List of Tables}
\else
  \newcommand\listtablename{List of Tables}
\fi
\ifdefined\figurename
  \renewcommand*\figurename{Figure}
\else
  \newcommand\figurename{Figure}
\fi
\ifdefined\tablename
  \renewcommand*\tablename{Table}
\else
  \newcommand\tablename{Table}
\fi
}
\@ifpackageloaded{float}{}{\usepackage{float}}
\floatstyle{ruled}
\@ifundefined{c@chapter}{\newfloat{codelisting}{h}{lop}}{\newfloat{codelisting}{h}{lop}[chapter]}
\floatname{codelisting}{Listing}
\newcommand*\listoflistings{\listof{codelisting}{List of Listings}}
\makeatother
\makeatletter
\@ifpackageloaded{caption}{}{\usepackage{caption}}
\@ifpackageloaded{subcaption}{}{\usepackage{subcaption}}
\makeatother
\makeatletter
\@ifpackageloaded{tcolorbox}{}{\usepackage[skins,breakable]{tcolorbox}}
\makeatother
\makeatletter
\@ifundefined{shadecolor}{\definecolor{shadecolor}{rgb}{.97, .97, .97}}
\makeatother
\makeatletter
\makeatother
\makeatletter
\makeatother
\ifLuaTeX
  \usepackage{selnolig}  % disable illegal ligatures
\fi
\IfFileExists{bookmark.sty}{\usepackage{bookmark}}{\usepackage{hyperref}}
\IfFileExists{xurl.sty}{\usepackage{xurl}}{} % add URL line breaks if available
\urlstyle{same} % disable monospaced font for URLs
\hypersetup{
  pdftitle={TRABAJO DE TITULACIÓN},
  pdfauthor={Bryan J. Aragón P.},
  colorlinks=true,
  linkcolor={blue},
  filecolor={Maroon},
  citecolor={Blue},
  urlcolor={Blue},
  pdfcreator={LaTeX via pandoc}}

\title{TRABAJO DE TITULACIÓN}
\usepackage{etoolbox}
\makeatletter
\providecommand{\subtitle}[1]{% add subtitle to \maketitle
  \apptocmd{\@title}{\par {\large #1 \par}}{}{}
}
\makeatother
\subtitle{Universidad Técnica de Ambato - Maestría en Economía}
\author{Bryan J. Aragón P.}
\date{2023-09-18}

\begin{document}
\maketitle
\ifdefined\Shaded\renewenvironment{Shaded}{\begin{tcolorbox}[breakable, borderline west={3pt}{0pt}{shadecolor}, interior hidden, sharp corners, boxrule=0pt, enhanced, frame hidden]}{\end{tcolorbox}}\fi

\renewcommand*\contentsname{Table of contents}
{
\hypersetup{linkcolor=}
\setcounter{tocdepth}{2}
\tableofcontents
}
\bookmarksetup{startatroot}

\hypertarget{tuxedtulo}{%
\chapter*{TÍTULO}\label{tuxedtulo}}
\addcontentsline{toc}{chapter}{TÍTULO}

\markboth{TÍTULO}{TÍTULO}

\begin{itemize}
\tightlist
\item
  \textbf{IMPACTO DE LA CONTRATACIÓN PÚBLICA EN LA ECONOMÍA ECUATORIANA,
  UN ANÁLISIS MEDIANTE EL APRENDIZAJE AUTOMÁTICO.}
\end{itemize}

\bookmarksetup{startatroot}

\hypertarget{luxednea-de-investigaciuxf3n}{%
\chapter*{LÍNEA DE INVESTIGACIÓN}\label{luxednea-de-investigaciuxf3n}}
\addcontentsline{toc}{chapter}{LÍNEA DE INVESTIGACIÓN}

\markboth{LÍNEA DE INVESTIGACIÓN}{LÍNEA DE INVESTIGACIÓN}

\begin{itemize}
\tightlist
\item
  Primera: Ciclo Económico y Sistema Financiero Nacional.
\end{itemize}

\begin{quote}
\emph{El proceso investigativo en esta línea se concentra en la
identificación de las fases del ciclo económico del territorio ,
alineada con la exploración de variables como consumo, inversión, empleo
,producción de bienes y servicios, la cual se desarrolla en función de
la actividad productiva de la zona en un tiempo determinado, (\ldots),
permite dilucidar con certeza la fase en la que se encuentra
económicamente el país, y alinearla con el desarrollo del sistema
financiero nacional, en cuanto hace referencia a la determinación de
políticas públicas resilientes, empáticas y honestas, del mismo modo
conocer e interpretar aspectos relevantes del Sistema Financiero
Nacional.}
\end{quote}

\begin{itemize}
\tightlist
\item
  Tercera: La política monetaria en una economía dolarizada y el sistema
  financiero nacional.
\end{itemize}

\begin{quote}
\emph{El desarrollo de esta línea pretende aportar la sostenibilidad y
transparencia de la gestión de las finanzas públicas, ya que, el
análisis de la política monetaria como objetivo central de controlar y
mantener la estabilidad económica, se relaciona con el tratamiento de
las finanzas públicas, de esta forma las decisiones de ingresos, egresos
de las instituciones de sector público, se suscriben hacia la política
pública, amerita del desarrollo técnico de las variable macroeconómicas
de monitoreo, control y evaluación del gasto público.}
\end{quote}

\bookmarksetup{startatroot}

\hypertarget{capuxedtulo-i}{%
\chapter{CAPÍTULO I}\label{capuxedtulo-i}}

\hypertarget{introducciuxf3n}{%
\section{INTRODUCCIÓN}\label{introducciuxf3n}}

\hypertarget{justificaciuxf3n}{%
\section{JUSTIFICACIÓN}\label{justificaciuxf3n}}

\hypertarget{formulaciuxf3n-del-problema-de-investigaciuxf3n}{%
\section{Formulación del problema de
investigación}\label{formulaciuxf3n-del-problema-de-investigaciuxf3n}}

¿Que impacto tiene el gasto de la contratación pública en la economía
Ecuatoriana durante el periodo 20xx -- 20xx?

\hypertarget{objetivos}{%
\section{OBJETIVOS}\label{objetivos}}

\hypertarget{general}{%
\subsection{GENERAL}\label{general}}

\begin{itemize}
\tightlist
\item
  Analizar el impacto de la contratación pública en la economía
  ecuatoriana mediante el uso de técnicas de aprendizaje automático, con
  el fin de identificar patrones, tendencias y relaciones que
  contribuyan a comprender cómo el gasto en contratación pública afecta
  el desarrollo económico del país.''
\end{itemize}

\hypertarget{especuxedficos}{%
\subsection{ESPECÍFICOS}\label{especuxedficos}}

\begin{itemize}
\item
  Detallar la evolución de la Contratación pública durante el periodo
  20xx - 20xx
\item
  Evaluar el impacto de la contratación pública en la economía
  ecuatoriana, considerando aspectos como el crecimiento del PIB, la
  generación de empleo, la inversión pública y la eficiencia en el gasto
  gubernamental, y proponer recomendaciones basadas en los resultados
  del análisis de aprendizaje automático para mejorar la toma de
  decisiones y la gestión de la contratación pública en el país.
\item
  Aplicar técnicas de aprendizaje automático, como el análisis de datos,
  la modelización predictiva, a los datos recopilados con el fin de
  identificar patrones, tendencias y relaciones entre los procesos de
  contratación pública y los indicadores económicos y financieros del
  país.
\end{itemize}

\begin{tcolorbox}[enhanced jigsaw, left=2mm, coltitle=black, colbacktitle=quarto-callout-note-color!10!white, bottomtitle=1mm, toprule=.15mm, bottomrule=.15mm, breakable, colframe=quarto-callout-note-color-frame, opacitybacktitle=0.6, toptitle=1mm, colback=white, rightrule=.15mm, titlerule=0mm, title=\textcolor{quarto-callout-note-color}{\faInfo}\hspace{0.5em}{Note}, leftrule=.75mm, arc=.35mm, opacityback=0]

Los objetivos se fundamentan dentro de una comprensión
\textbf{holística}, la investigación tiene como propósito la búsqueda y
generación de conocimiento, de modo tal que ese conocimiento pueda tener
diferentes grados de elaboración (Exploraciones, descripciones,
análisis, comparaciones, explicaciones, predicciones, transformaciones,
verificaciones, evaluaciones)

\end{tcolorbox}

\begin{tcolorbox}[enhanced jigsaw, left=2mm, coltitle=black, colbacktitle=quarto-callout-note-color!10!white, bottomtitle=1mm, toprule=.15mm, bottomrule=.15mm, breakable, colframe=quarto-callout-note-color-frame, opacitybacktitle=0.6, toptitle=1mm, colback=white, rightrule=.15mm, titlerule=0mm, title=\textcolor{quarto-callout-note-color}{\faInfo}\hspace{0.5em}{Note}, leftrule=.75mm, arc=.35mm, opacityback=0]

1 verbo en infinitivo, 2 objeto de estudio, 3 como lo hago, 4 para que
(justificación).

\end{tcolorbox}

\begin{tcolorbox}[enhanced jigsaw, left=2mm, coltitle=black, colbacktitle=quarto-callout-note-color!10!white, bottomtitle=1mm, toprule=.15mm, bottomrule=.15mm, breakable, colframe=quarto-callout-note-color-frame, opacitybacktitle=0.6, toptitle=1mm, colback=white, rightrule=.15mm, titlerule=0mm, title=\textcolor{quarto-callout-note-color}{\faInfo}\hspace{0.5em}{Note}, leftrule=.75mm, arc=.35mm, opacityback=0]

Entrenar y evaluar un modelo de regresión con diferentes algoritmos.
Realizar un preprocesamiento de datos. Ajustar algunos hiperparámetros.
Realizar mejores predicciones.

\end{tcolorbox}

\begin{tcolorbox}[enhanced jigsaw, left=2mm, coltitle=black, colbacktitle=quarto-callout-note-color!10!white, bottomtitle=1mm, toprule=.15mm, bottomrule=.15mm, breakable, colframe=quarto-callout-note-color-frame, opacitybacktitle=0.6, toptitle=1mm, colback=white, rightrule=.15mm, titlerule=0mm, title=\textcolor{quarto-callout-note-color}{\faInfo}\hspace{0.5em}{Note}, leftrule=.75mm, arc=.35mm, opacityback=0]

totalCP, totalDeIngresosDelSP TotalGastosDelSP, inflación, Empleo
CanastaFamiliar, SaldoDeLaDeudaPublicaInterna, ÍNDICE DE PRECIOS AL
PRODUCTOR DE DISPONIBILIDAD NACIONAL, LIQUIDEZ TOTAL (M2), balanza
comercial (TOMAR EN CUENTA)

\end{tcolorbox}

\begin{Shaded}
\begin{Highlighting}[]
\DecValTok{2}\SpecialCharTok{*}\DecValTok{2}
\end{Highlighting}
\end{Shaded}

\begin{verbatim}
[1] 4
\end{verbatim}

\begin{Shaded}
\begin{Highlighting}[]
\DecValTok{2}\SpecialCharTok{*}\DecValTok{2}
\end{Highlighting}
\end{Shaded}

\begin{verbatim}
[1] 4
\end{verbatim}

\bookmarksetup{startatroot}

\hypertarget{capuxedtulo-ii}{%
\chapter{CAPÍTULO II}\label{capuxedtulo-ii}}

In summary, this book has no content whatsoever.

\begin{Shaded}
\begin{Highlighting}[]
\DecValTok{1} \SpecialCharTok{+} \DecValTok{1}
\end{Highlighting}
\end{Shaded}

\begin{verbatim}
[1] 2
\end{verbatim}

\bookmarksetup{startatroot}

\hypertarget{capuxedtulo-iii}{%
\chapter{CAPÍTULO III}\label{capuxedtulo-iii}}

\bookmarksetup{startatroot}

\hypertarget{references}{%
\chapter*{References}\label{references}}
\addcontentsline{toc}{chapter}{References}

\markboth{References}{References}

\hypertarget{refs}{}
\begin{CSLReferences}{0}{0}
\end{CSLReferences}



\end{document}
